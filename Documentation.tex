\documentclass{sig-alternate}
\usepackage{auto-pst-pdf}

\begin{document}

\title{Automatic User Profiling for Intelligent
Tourist Trip Personalisation}
\numberofauthors{2} %  in this sample file, there are a *total*
\author{
% 1st. author
\alignauthor
Liam Attard\\
       \affaddr{University of Malta}\\
       \affaddr{Msida, Malta}\\
       \email{liam.attard.18@um.edu.mt}
% 2nd. author
\alignauthor
Dr Josef Bajada\\
       \affaddr{University of Malta}\\
       \affaddr{Msida, Malta}\\
       \email{josef.bajada@um.edu.mt}
}

\date{12th July 2021}

\makeatletter
\def\@copyrightspace{\relax}
\makeatother

\maketitle
\begin{abstract}
    The objective of holiday activity planning is to maximise the
    traveller's enjoyment during such trips by selecting the right places to
    visit and things to do according to the person's preferences. This process
    involves preparing information from various data sources, which is often
    very time-consuming. This project presents a tourist itinerary
    recommendation algorithm that assists users by autonomously generating a
    personalised holiday plan according to the user's travel dates and
    constraints.  Furthermore, the system automatically builds a travel
    interest profile from the user's social media presence, which is then used
    to recommend itineraries tailored to the user's interests. The system uses
    social media APIs from popular platforms such as Facebook and Instagram.
    With the user's permission, the system gathers information such as pages
    the user likes and pictures posted by the user. 

    A Convolution Neural Network  is used to classify the user's pictures into their respective
    travel category, such as Beach, Clubbing, Nature, Museums or Shopping,
    which is then used to determine the user's predominant travel interest
    topics. A Resnet-18, Resnet-50 and Keras Sequential model are validated 
    separately on a testing dataset to see which one works best.  This computed
    travel profile of a user takes the form of a weight vector, which is then
    used to generate an automated itinerary that fits the user's preferences
    and travel constraints. 

    This weight vector is used to formulate a
    personalised objective function used by various meta-heuristic and
    evolutionary algorithms to optimise the plan. The algorithms consider hard
    constraints such as holiday dates, distances between places, and soft
    constraints (preferences), such as the interests and the user's preferred
    pace. This dissertation compares Particle Swarm Optimisation and Genetic
    Algorithms, and they are evaluated for both their plan quality and
    performance.

    Since the results are highly personalised, the system was packaged into an
    application that allows users to connect with their social media accounts,
    build a personalised travel plan for a holiday in Malta, and ask the user to
    assess the plan's quality with respect to personal preferences and activity pace. The
    user is also asked to assess a more generic holiday itinerary without
    specification of the generated plan, in order to assess the
    effectiveness of the personalised holiday planning algorithm. 

\end{abstract}

% A category with the (minimum) three required fields
\category{H.4}{Information Systems Applications}{Miscellaneous}
%A category including the fourth, optional field follows...
\category{D.2.8}{Software Engineering}{Metrics}[complexity measures, performance measures]

\terms{Theory}

\keywords{ACM proceedings, \LaTeX, text tagging}

\section{Introduction}
\section{The {\secit Body} of The Paper}
\subsection{Citations}
Citations to articles \cite{bowman:reasoning,
clark:pct, braams:babel, herlihy:methodology},
conference proceedings \cite{clark:pct} or
books \cite{salas:calculus, Lamport:LaTeX} listed
in the Bibliography section of your
article will occur throughout the text of your article.
You should use BibTeX to automatically produce this bibliography;
you simply need to insert one of several citation commands with
a key of the item cited in the proper location in
the \texttt{.tex} file \cite{Lamport:LaTeX}.
The key is a short reference you invent to uniquely
identify each work; in this sample document, the key is
the first author's surname and a
word from the title.  This identifying key is included
with each item in the \texttt{.bib} file for your article.

The details of the construction of the \texttt{.bib} file
are beyond the scope of this sample document, but more
information can be found in the \textit{Author's Guide},
and exhaustive details in the \textit{\LaTeX\ User's
Guide}\cite{Lamport:LaTeX}.

This article shows only the plainest form
of the citation command, using \texttt{{\char'134}cite}.
This is what is stipulated in the SIGS style specifications.
No other citation format is endorsed or supported.

\section{Conclusions}
This paragraph will end the body of this sample document.
Remember that you might still have Acknowledgments or
Appendices; brief samples of these
follow.  There is still the Bibliography to deal with; and
we will make a disclaimer about that here: with the exception
of the reference to the \LaTeX\ book, the citations in
this paper are to articles which have nothing to
do with the present subject and are used as
examples only.

\section{Acknowledgments}
This section is optional; it is a location for you
to acknowledge grants, funding, editing assistance and
what have you.  In the present case, for example, the
authors would like to thank Gerald Murray of ACM for
his help in codifying this \textit{Author's Guide}
and the \textbf{.cls} and \textbf{.tex} files that it describes.

%
% The following two commands are all you need in the
% initial runs of your .tex file to
% produce the bibliography for the citations in your paper.
\bibliographystyle{abbrv}
\bibliography{sigproc}  % sigproc.bib is the name of the Bibliography in this case
% You must have a proper ".bib" file
%  and remember to run:
% latex bibtex latex latex
% to resolve all references
%
% ACM needs 'a single self-contained file'!
%
%APPENDICES are optional
%\balancecolumns
\appendix
%Appendix A
\section{Headings in Appendices}
The rules about hierarchical headings discussed above for
the body of the article are different in the appendices.
In the \textbf{appendix} environment, the command
\textbf{section} is used to
indicate the start of each Appendix, with alphabetic order
designation (i.e. the first is A, the second B, etc.) and
a title (if you include one).  So, if you need
hierarchical structure
\textit{within} an Appendix, start with \textbf{subsection} as the
highest level. Here is an outline of the body of this
document in Appendix-appropriate form:
\subsection{Introduction}
\subsection{The Body of the Paper}
\subsubsection{Type Changes and  Special Characters}
\subsubsection{Math Equations}
\paragraph{Inline (In-text) Equations}
\paragraph{Display Equations}
\subsubsection{Citations}
\subsubsection{Tables}
\subsubsection{Figures}
\subsubsection{Theorem-like Constructs}
\subsubsection*{A Caveat for the \TeX\ Expert}
\subsection{Conclusions}
\subsection{Acknowledgments}
\subsection{Additional Authors}
This section is inserted by \LaTeX; you do not insert it.
You just add the names and information in the
\texttt{{\char'134}additionalauthors} command at the start
of the document.
\subsection{References}
Generated by bibtex from your ~.bib file.  Run latex,
then bibtex, then latex twice (to resolve references)
to create the ~.bbl file.  Insert that ~.bbl file into
the .tex source file and comment out
the command \texttt{{\char'134}thebibliography}.
% This next section command marks the start of
% Appendix B, and does not continue the present hierarchy
\section{More Help for the Hardy}
The sig-alternate.cls file itself is chock-full of succinct
and helpful comments.  If you consider yourself a moderately
experienced to expert user of \LaTeX, you may find reading
it useful but please remember not to change it.
%\balancecolumns % GM June 2007
% That's all folks!
\end{document}
