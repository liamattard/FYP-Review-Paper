

\subsection{Producing the activity plan}

The problem of our itinerary planner algorithm
is mathematically formulated as follows. A tourist trip is made up
of some pre-defined user constants alongside the travel interest
vector. The predefined constants are:
\\
\setlength{\tabcolsep}{20pt}

\begin{tabular}{l l}

\textit{M}:  &  The number of travelling days. \\
\textit{C}: & The activity pace.\\  
\end{tabular}
\\
\\
The objective function of our itinerary planner is:

\[ \text{MAX}  \sum_{m=0}^{M} ( S_{{D_m}} + S_{{E_m}}) \]
where:
\\
\begin{tabular}{l l}
% \textit{i,j}:  &  POI (\textit{i,j} = 2,3,...,\textit{N}) \\
\textit{m} & Travelling day (\textit{m}=1,2,\ldots, \textit{M}) \\ 
\textit{$D_m$} & Morning section of day number m \\  
\textit{$E_m$} & Evening section of day number m \\  
\textit{$S_{D_m}$} & Score of the morning section $D_m$ \\  
\textit{$S_{E_m}$} & Score of the evening section $E_m$ \\  
\end{tabular}
\\
\\

A day is made up of the morning ${D_m}$ section and
the evening ${E_m}$ section. The timetable suggests a
POI in the morning, then somewhere to eat, and the
rest is dependant on the activity pace C. That is why
the morning section is made up of $C + 2$ tourist
attractions. The evening section suggests a place to
eat and a POI; therefore, the evening section is just
made up of $2$. 

\begin{center}
$D_m = Y_i + Y_f + C ( Y_i)$ and $E_m = Y_f + Y_j $

\end{center}



\begin{tabular}{l l}

\textit{i} & Morning attraction (i = 1,…, $n_1$)\\
\textit{j} & Evening attraction (j = 1,…, $n_2$)\\
\textit{f} & Food Place (f = 1,…, $n_3$)\\
\textit{$Y_{i|f|j}$}: & Number of times POI is visited. \\
\end{tabular}
\\ 
\\
\begin{tabular}{l l}
\textbf{Constraints} & \\
\textit{$ \sum_{m=0}^{M}\sum_{i=0}^{n_1}{Y_i} \leq 1$} & Ensures morning POIs\\ & not visited more than once. \\

\textit{$ \sum_{m=0}^{M}\sum_{j=0}^{n_1}{Y_j} \leq 1$} & Ensures evening POIs\\ & not visited more than once.\\


\end{tabular}

The score $S_{D_m}$ or $S_{E_m}$ is calculated using

\[ S_{D_m | E_m} = \frac{1}{T} + R + V\]

where:
\\
\begin{tabular}{l l}
% \textit{i,j}:  &  POI (\textit{i,j} = 2,3,...,\textit{N}) \\
\textit{T} & Distance between POIs of day m\\ 
\textit{R} & Average rating of POIs of day m\\  
\textit{V} & How much POIs match with the \\ &  user's travel interest vector. \\  

\end{tabular}
\\



%\input{sections/methodology/travelProducts.tex}

%\subsubsection{}
\subsubsection{Optimisation Algorithms}

PSO and GAs are two meta-heuristics that use a population
to converge to a fit solution. Therefore, they require
an initial random generation of possible timetables. In
our algorithm, we introduce a method of randomisation
bias. With this technique, the randomness of the initial
population is weighted based on the place's rating and 
the place's number of ratings. This bias gives a 
head start to the algorithm rather than just starting 
optimising from purely random itineraries, highly likely 
to be of bad quality.

%\begin{algorithm}[h]

 iter = x \;
 count = 0 \;
 particles = RandomBiasInitialiser(); \;
 bestParticle = particle[0] \;

 \While{count $>$ x}{

     i = 0 \;
     maxScore = 0 \;

     \While{i $<=$ particles.length}{

         \tcc{Updating personal best}
        \If{particles[i].score $>$ particles[i].personalBest.score}{

             particles[i].personalBest = particles[i].position\;

         }

         \tcc{Updating global best}
        \If{particles[i].score $>$ maxScore}{


             maxScore = particles[i].score \;
             bestParticle = particles[i].position \;

         }

         i = i +1 \;

         \tcc{Calculate new position}
         particle[i].calculateNewPosition() \;



     }
     \Return bestParticle

 }

 \caption{Particle Swarm Optimisation}
    \label{PSOAlgorithm}
\end{algorithm}




\paragraph{Particle Swarm Optimisation}

In PSO, the whole population is referred to as the
swarm, whilst a single member a particle. Each
particle has a 2-dimensional position(P) vector
representing the current timetable solution and a
2-dimensional velocity(V) vector expressing the direction
of the particle during its search period. 

The algorithm has six integer parameters including the
number of particles and the number of iterations. The
personal acceleration (PA) affects how far away the
particle moves from the personal best position (PB). 
The global best acceleration (GA) attracts the global
best position (GB) of the whole swarm. 
The inertia (I) constant helps the particle
explore new solutions and escape the local minima
through randomness. 

At each
iteration, the new velocity is calculated using
\begin{center}
    new velocity = I + (PA * (PB - P)) + (GA * (GB - P))
\end{center}

the new position is calculated using
\begin{center}
    new position = P + new velocity 
\end{center}

After a few iterations have passed, particles use
their velocity and move towards the optimum position.
We demonstrate the framework of our PSO algorithm in
algorithm \ref{PSOAlgorithm}.


\paragraph{Genetic Algorithms}

Genetics algorithms use biological terms to describe
their attributes. For example, a timetable solution in
population is referred to as a chromosome~\ref{Abbaspour2011}. 

In PSO, the algorithm optimises by allowing each
particle to move closer to the global best every
iteration. In comparison, in GAs, first, the best
chromosomes known as the elites are selected from each
iteration. Then, three techniques, namely selection,
mutation, and crossover, are applied to generate the
next population.

We used the geneticalgorithm2 package~\footnote{https://pypi.org/project/geneticalgorithm2/},
which allowed us to use the same score function and
the random bias to initialise the particles. The
algorithm has 7 parameters. The parents portion
represents the number of parents who will reproduce
and create the next generation. The mutation
probability determines the chance a POI in a
chromosome will be replaced by a random value to which
the algorithm will converge less quickly and explore
more of the search space. The crossover probability
will affect the chance that part of its solution goes
to the child. Finally, the elite ratio determines how
much of the best chromosomes in an iteration make it
to the next iteration.  There are many types of
crossovers techniques. In this algorithm, we
explored \textit{one point, two point, uniform and shuffle crossover}.  The algorithm produced the following
steps: \\
\textbf{Step 1}: Initialise the first population using random bias. \\
\textbf{Step 2}: Select the best chromosomes from the population. \\ 
\textbf{Step 3}: Select the elite particles that will make it to the next iteration.\\
\textbf{Step 3}: Apply Crossover, Mutation and Selection on the population. \\
\textbf{Step 4}: Check if the number of iterations has exceeded. 


